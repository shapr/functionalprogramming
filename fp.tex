% Shae Erisson <serisson@una.edu>
% University of North Alabama
% 09/27/12
% ghclive

%% TODO: remove text from picture slides, just say that?
%% XXXintroduce haskell and irc earlier, definitions slide
%% big picture of what haskell is, what the environment is
% hpaste, lambdabot eval?
%% sooner, faster, earlier!
% more time talking about Haskell
% google docs handles real-time collab, but not source code
% user feedback:
% people would like to see the cursor location for collaborators
% people would like multiple tabs for multiple source files
% people want this for Python, Ruby and C++ at least
% if you do research on humans: human subjects board - irb

%% stretch pictures so text in pix is slide sized!
%% GSoC slide should just have numbers, text should be my presenter notes
%% be VERY clear about the problem: helping novice programmers learn haskell across text chat
%% slow down by 50%, this is new information to most listeners

%% I want people to take away from this: my solution to the problem of teaching across the 'net
%% is a good solution, and can be generalized to other programming languages

\documentclass[xcolor=pdftex,dvipsnames,table]{beamer}
% Do NOT load xcolor and graphicx, beamer already loads these
\usepackage{array}
\usepackage{multirow}
\usepackage{url}
\usepackage{haskelllogo}
\usepackage{minted}
% Set theme to UNA
\usetheme{UNA}

% ##############################################################################
% Add "handout" to the documentclass options
% Uncomment the following line for handouts
%\usepackage{pgfpages}
% Uncomment 1 of the 2 following lines for handouts
%\pgfpagesuselayout{4 on 1}[letterpaper,landscape,border shrink=5mm]
%\pgfpagesuselayout{2 on 1}[letterpaper,border shrink=5mm]
% ##############################################################################

% Title page information
\title[University of North Alabama]{Functional Programming Languages}
\author[CS470 Artificial Intelligence]{Andrew Aaron,Shae Erisson,PengCheng Zhao}
\institute[University of North Alabama]{Department of Computer Science and Information Systems\\University of North Alabama}
\date{April 15, 2013}

\begin{document}

% Make a title frame
\begin{frame}
  \maketitle
\end{frame}

% Make an automatic outline frame, relies on sections and subsections
%% \begin{frame}
%%   \frametitle{Presentation outline}
%%   % \tableofcontents
%%   \setbeamercovered{dynamic}
%%   \begin{itemize}
%%   \item Background
%%     \pause
%%   \item Problem
%%     \pause
%%   \item Proposal
%%     \pause
%%   \item Conclusion
%%   \end{itemize}
%% \end{frame}

\section{Introduction}
\subsection{What will we cover?}
\begin{frame} % \subsection{Google Summer of Code}
  \frametitle{What will we cover?}
  \setbeamercovered{dynamic}
  \begin{itemize}
  \item What is functional programming?
  \item Why is it useful for AI?
  \item What does it look like?
  \item Where do I get one?
  \item How can I benefit from all this awesome?
  \end{itemize}
\end{frame}

\subsection{What should you know after these two classes?}
\begin{frame}
  \frametitle{What should you know after these two classes?}
  \begin{itemize}
  \item What's good and bad about functional programming
  \item How it might benefit you
  \end{itemize}
\end{frame}

\section{What is Functional Programming?}
\subsection{Definitions}
\begin{frame}
  \frametitle{Definitions}
  \begin{itemize}
  \item function - as in math
  \item functional - functions as first-class values
  \item purely functional - lack of side effects
  \end{itemize}
\end{frame}

\subsection{What is it good for?}
\begin{frame}
  \frametitle{What is it good for?}
  \begin{itemize}
  \item Math
  \item Algorithms
  \item Prototyping
  \end{itemize}
\end{frame}

\subsection{Who uses it?}
\begin{frame}
  \frametitle{Who uses it?\footnote{\url{http://www.haskell.org/haskellwiki/Haskell_in_industry}}}
  \begin{itemize}
  \item Facebook
  \item Google
  \item AT\&T
  \item NVidia
  \item Alcatel-Lucent
  \end{itemize}
\end{frame}

\subsection{What makes it better?}
\begin{frame}
  \frametitle{What makes it better?}
  \begin{itemize}
  \item Garbage collection, no memory leaks!
  \item Works like an equation
  \item implicit data traversal simplifies loops
  \item first class functions make code easier to write and read
  \item Requires less time to get a working program
  \end{itemize}
\end{frame}

\subsection{What makes it worse?}
\begin{frame}
  \frametitle{What makes it worse?}
  \begin{itemize}
  \item Garbage collection costs more CPU cycles
  \item Requires more memory to compile and run
  \item Haskell/Scheme is not C++, or even close
  \end{itemize}
\end{frame}

\section{Why is functional programming useful for AI?}
\subsection{Why is it useful for AI?}
\begin{frame}
  \frametitle{Why is it useful for AI?}
  \begin{itemize}
  \item Working programs produced inside research budgets
  \item Academics favor equational reasoning
  \item lack of side effects eases composition
  \end{itemize}
\end{frame}

\subsection{How is functional different from imperative?}
\begin{frame}
  \frametitle{How is functional different from imperative?}
  \begin{itemize}
  \item functions are values
  \item purely functional means variables don't vary
  \item recursion instead of loops
  \end{itemize}
\end{frame}

\subsection{Could this make your homework easier?}
\begin{frame}
  \frametitle{Could this make your homework easier?}
  \begin{itemize}
  \item Can these features simplify your C++ homework?
  \item Yes, I mean you in the back row!
  \end{itemize}
\end{frame}

\section{What does it look like?}
\subsection{What does it look like?}
\begin{frame}
  \frametitle{What does one look like?}
  \mint{haskell}|factorial n = product [1..n]|
  \inputminted{scheme}{fac.scm}
\end{frame}


\subsection{What's unique about Haskell?}
\begin{frame}
  \frametitle{What's unique about Haskell?}
  \begin{itemize}
  \item \textbf{purely} functional
  \item laziness
  \item type system
  \item data types
  \end{itemize}
\end{frame}

% good demo for macros:
% http://mebdev.blogspot.com/2011/06/brief-intro-to-writing-macros-in-racket.html
\subsection{What's unique about Scheme?}
\begin{frame}
  \frametitle{What's unique about Scheme?}
  \begin{itemize}
  \item data and programs are the same
  \item excessive use of lists
  \end{itemize}
\end{frame}

\section{Where do I get one?}
\subsection{Where can I download Haskell and Scheme?}
\begin{frame}
  \frametitle{Where can I download Haskell and Scheme?}
  \begin{itemize}
  \item Haskell - \url{http://www.haskell.org}
  \item Scheme - \url{http://www.racket-lang.org}
  \end{itemize}
\end{frame}

\subsection{Where do I get tutorials?}
\begin{frame}
  \frametitle{Where do I get tutorials?}
  \begin{itemize}
  \item Haskell - \url{http://www.haskell.org/haskellwiki/Learning_Haskell}
  \item Scheme - \url{http://docs.racket-lang.org/getting-started/}
  \end{itemize}
\end{frame}

\section{Homework!}
\begin{frame}
  \frametitle{Homework!}
  Given a tree and code to do a depth-first-search, can you write the breadth-first-search?
  \begin{itemize}
  \item Haskell - \url{http://www.ScannedInAvian.com/~shae/haskellsearch.zip}
  \item Scheme - \url{http://www.ScannedInAvian.com/~shae/schemesearch.zip}
  \end{itemize}
\end{frame}

\subsection{Questions?}
\begin{frame}[plain]
  \label{thanks}
  \frametitle{Questions?}
  \begin{center}
    \textbf{Thank you!}
  \end{center}
  \begin{itemize}
  \item http://github.com/shapr/ghclive
  \item http://ghclive.wordpress.com
  \end{itemize}
  \vspace{0.5cm}
  \begin{center}
    \begin{minipage}{11cm}
      \begin{columns}[c]
        \column{.2\textwidth}
        % \includegraphics[scale=.25]{faculty-web}
        \column{.8\textwidth}
        Shae Erisson\\
        Dept. of Computer Science \& Information Systems\\
        University of North Alabama\\
        \url{http://www.ScannedInAvian.com/}
      \end{columns}
    \end{minipage}
  \end{center}
\end{frame}


%\begin{frame}
%  \begin{center}
%    Thank You
%  \end{center}
%\end{frame}

\end{document}
